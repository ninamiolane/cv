% Co-organizer - Two Datathons (Global and University) in Collaboration with Kaggle and Women in Datascience (WIDS) (\url{www.widsworldwide.org/learn/datathon}) - (2025)

% Lead developer - TopoNetX, TopoEmbedX and TopoModelX (\url{github.com/pyt-team})\\

% Lead organizer - Open-Source Coding Challenges , \url{pyt-team.github.io/topomodelx/challenge/index.html}, \url{pyt-team/challenge-icml-2024})\\
        


\begin{tabularx}{\linewidth}{ @{}l r@{} }
\textbf{Geomstats} & \hfill \href{https://github.com/geomstats/geomstats}{GitHub} \\[3.75pt]
\multicolumn{2}{@{}X@{}}{
\vspace{-3mm}
\begin{itemize}[leftmargin=*,itemsep=0.em,topsep=0.em,parsep=0pt,partopsep=0pt]
    \item Creator, developer, and organizer of international hackathons
    \item Organizer of ICLR Computational Geometry \& Topology Challenge in \href{https://github.com/geomstats/challenge-iclr-2021}{2021} and \href{https://github.com/geomstats/challenge-iclr-2022}{2022}
\end{itemize}
}
\\
\end{tabularx}

\begin{tabularx}{\linewidth}{ @{}l r@{} }
\textbf{TopoX} & \hfill \href{https://pyt-team.github.io/}{GitHub} \\[3.75pt]
\multicolumn{2}{@{}X@{}}{
\vspace{-3mm}
\begin{itemize}[leftmargin=*,itemsep=0.em,topsep=0.em,parsep=0pt,partopsep=0pt]
    \item Co-creator and developer for packages \href{https://github.com/pyt-team/TopoNetX}{TopoNetX}, \href{https://github.com/pyt-team/TopoEmbedX}{TopoEmbedX} and \href{https://github.com/pyt-team/TopoModelX}{TopoModelX} 
    \item Organizer of the ICML Topological Deep Learning Challenge in \href{https://pyt-team.github.io/topomodelx/challenge/index.html}{2023} and \href{https://pyt-team.github.io/packs/challenge.html}{2024}
\end{itemize}
}
\\
\end{tabularx}

\begin{tabularx}{\linewidth}{ @{}l r@{} }
\textbf{TopoBench} & \hfill \href{https://github.com/geometric-intelligence/TopoBench}{GitHub} \\[3.75pt]
\multicolumn{2}{@{}X@{}}{
\vspace{-3mm}
\begin{itemize}[leftmargin=*,itemsep=0.em,topsep=0.em,parsep=0pt,partopsep=0pt]
    \item Developer and senior mentor
    \item Senior mentor for the TAG 2025 Topological Deep Learning Challenge
\end{itemize}
}
\\
\end{tabularx}


% \begin{tabularx}{\linewidth}{ @{}l r@{} }
% \textbf{Topological Deep Learning} & \hfill \href{https://geometric-intelligence.github.io/TopoBench/}{Link to Demo} \\[3.75pt]
% \multicolumn{2}{@{}X@{}}{The natural world is full of complex systems characterized by intricate relations between their components: from social interactions between individuals in a social network to interactions between neurons in a neural networks. Topological Deep Learning (TDL) provides a comprehensive framework to process and extract knowledge from data associated with these systems.}  \\
% \end{tabularx}

% \begin{tabularx}{\linewidth}{ @{}l r@{} }
% \textbf{Equivariant Deep Learning} & \hfill \href{https://github.com/geometric-intelligence/g-invariance}{Link to Demo} \\[3.75pt]
% \multicolumn{2}{@{}X@{}}{Many of the transformations that occur in images are due to the actions of groups: groups of translations, rotations, scalings, among others. Group-equivariant neural networks are special types of neural networks that preserve the structure of these transformations. We say that they respect the symmetries of the data.}  \\
% \end{tabularx}

