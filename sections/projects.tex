\begin{tabularx}{\linewidth}{ @{}l r@{} }
\textbf{Neural Manifolds} & \hfill \href{https://geometric-intelligence.github.io/neurometry/}{Link to Demo} \\[3.75pt]
\multicolumn{2}{@{}X@{}}{How do brains and AI models structure information? We explore the geometric representations of intelligence, by quantifying the dimensions, topologies, and curvatures of neural manifolds, i.e., the smooth spaces of neuronal activity found in neuronal circuits such as head direction circuits, grid cells, place cells, and areas of visual cortex, and in artificial neural networks layers.}  \\
\end{tabularx}

\begin{tabularx}{\linewidth}{ @{}l r@{} }
\textbf{Topological Deep Learning} & \hfill \href{https://geometric-intelligence.github.io/TopoBench/}{Link to Demo} \\[3.75pt]
\multicolumn{2}{@{}X@{}}{The natural world is full of complex systems characterized by intricate relations between their components: from social interactions between individuals in a social network to interactions between neurons in a neural networks. We develop Topological Deep Learning (TDL) as a comprehensive framework to process and extract knowledge from data associated with these systems.}  \\
\end{tabularx}

\begin{tabularx}{\linewidth}{ @{}l r@{} }
\textbf{Equivariant Deep Learning} & \hfill \href{https://github.com/geometric-intelligence/g-invariance}{Link to Demo} \\[3.75pt]
\multicolumn{2}{@{}X@{}}{The structures we encounter in data are often governed by hidden symmetries: from the rotational invariance of molecules to the translational patterns of signals. These symmetries can be formalized through mathematical groups, which capture transformations such as shifts, rotations, or rescalings. We develop group-equivariant and \textit{complete} group-invariant neural networks as a principled framework to quotient out these symmetries, while retaining the richness of its other structural information.}  \\
\end{tabularx}

