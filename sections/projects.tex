\begin{tabularx}{\linewidth}{ @{}l r@{} }
\textbf{Neural Manifolds} & \hfill \href{https://geometric-intelligence.github.io/neurometry/}{Link to Demo} \\[3.75pt]
\multicolumn{2}{@{}X@{}}{How does our brain structure information? We explore the geometric representations of cognition, by quantifying the dimensions, topologies, and curvatures of neural manifolds, i.e. the smooth spaces of neuronal activity found in neuronal circuits such as head direction circuits, grid cells, place cells, and areas of visual cortex.}  \\
\end{tabularx}

\begin{tabularx}{\linewidth}{ @{}l r@{} }
\textbf{Topological Deep Learning} & \hfill \href{https://geometric-intelligence.github.io/TopoBench/}{Link to Demo} \\[3.75pt]
\multicolumn{2}{@{}X@{}}{The natural world is full of complex systems characterized by intricate relations between their components: from social interactions between individuals in a social network to interactions between neurons in a neural networks. Topological Deep Learning (TDL) provides a comprehensive framework to process and extract knowledge from data associated with these systems.}  \\
\end{tabularx}

\begin{tabularx}{\linewidth}{ @{}l r@{} }
\textbf{Equivariant Deep Learning} & \hfill \href{https://github.com/geometric-intelligence/g-invariance}{Link to Demo} \\[3.75pt]
\multicolumn{2}{@{}X@{}}{Many of the transformations that occur in images are due to the actions of groups: groups of translations, rotations, scalings, among others. Group-equivariant neural networks are special types of neural networks that preserve the structure of these transformations. We say that they respect the symmetries of the data.}  \\
\end{tabularx}

